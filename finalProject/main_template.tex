\documentclass[letterpaper, onecolumn,10pt]{IEEEtran}

\usepackage{graphicx}
\usepackage{amssymb}
\usepackage{amsmath}
\usepackage{amsthm}

\usepackage{alltt}
\usepackage{float}
\usepackage{color}
\usepackage{url}
\usepackage{listings}

\usepackage[TABBOTCAP, tight]{}

\usepackage{geometry}
\geometry{textheight=8.5in, textwidth=6in}

%random comment

\newcommand{\cred}[1]{{\color{red}#1}}
\newcommand{\cblue}[1]{{\color{blue}#1}}

\usepackage{hyperref}
\usepackage{geometry}
\usepackage{caption}
\usepackage{url}
\usepackage{natbib}

\begin{document}
    \begin{titlepage}
    \newcommand{\HRule}{\rule{\linewidth}{0.5mm}}
    \center
    \textsc{\Large Oregon State University}\\[1.5cm]
    \textsc{\Large CS 373}\\[0.5cm]
    \textsc{\Large Winter 2019}\\[0.5cm]
    \HRule \\[0.4cm]
    { \huge \bfseries Final Project}\\[0.4cm] % Title of your document
    \HRule \\[1.5cm]
    \begin{minipage}{0.4\textwidth}
        \begin{flushleft} \large
        \emph{Author:}\\
        Thomas Noelcke
        \end{flushleft}
    \end{minipage}
    \begin{minipage}{0.4\textwidth}
        \begin{flushright} \large
        \emph{Instructor:} \\
        D. Kevin McGrath\\
        \end{flushright}
    \end{minipage}\\[2cm]
		\end{titlepage}
		
		
		\section{Overview}
		In this write up I will talk about the Hack the Box challenge. I will describe how I obtained an invite code and registered. I will then discuss the three challenges that I chose to complete. In each section I will discuss the steps I took to complete the challenge including figures and code samples where applicable.\\
			
		\section{Obtaining an Invite Code}
		To start this project I visited the hack the box website. Once at the website I found the registration site for individuals. The page indicated that I would need to hack the page in order to obtain an invite code. The first thing I did was start examining the sources for the web page. I then noticed that there was file called inviteapi.min.js. This is a minified JS file.\\
		
		Using the dev tools on the browser I viewed this file. Initially, It was only five lines long and hard to look at. Using the dev tools I pretty printed the file to make the file more readable. I then noticed that it looked like this file was exporting some functions. The frist thing I tried to do was to evaluate this code in the console. This however gave me an undefined result. However, One of the functions that I noticed that this file was generating was makeInviteCode(). I looked at the arguments for this function using the console and found that this function didn't take any arguments. I called this function and it returned an object much like the object below.\\
		
		I noticed that this object had a odd looking string along with a compression algorithm. In this case it was base64. I took this string and decoded it using base64 and found that it was a message. The message stated: "In order to generate the invite code, make a POST request to /api/invite/generate". Next I used post man to make a post request to the end point that it suggested. Again I got a compressed string. I decoded this string also using base64 and was given an activation code kind of like a windows activation code. I entered this into the activation code box and found that it took me to the next screen to register.\\
		
	    \subsection{Challenges}
	        \subsubsection{Challenge 1: Snake}
	        The first challenge that I choose was snake. This was a reversing challenge. I choose this challenge first because I am a software developer and write code and debug code quite a bit. I figured that this would be a good starting challenge. This challenge was worth 10 points.\\
	        
	        To get started a read through the script and ensured that it wasn't doing any thing malicious. After inspecting the script I ran the script. I noticed that it asked for a user name. The user name was encoded in hex in the program by appending different character variables using hex. Decoding the username was easy.\\
	        
	        The password on the other hand was a bit more difficult to decode the password I first had to wade through the many unused variables and determine what was actually being used. Once I determined was was actually being used to create the password variable I decided to debug the program using VScode. Once I did this I was easily able to find the password and submitted the answer to hack the box.\\
	        
	        \subsubsection{Challenge 2: Lernaean}
	        The second challenge that I chose was Lernean. This challenge was worth 20 points and was more challenging than the first one. The goal of this challenge was to break into a site that some one had propped up and get the secret code. Supposedly, the person who set this site up isn't very good with computers.\\
	        
	        The first thing I did was go to the sight. The first thing I noticed is that the page said Please don't guess our password. Naturally, The first thing I tried was to guess the password. This however got old and I chose to take a more automated approach to solving this. I then went about setting up hydra on my machine and getting a rather large list of passwords from a github source. With these tools in place I was set up to blast the site with a large combination of passwords. After hitting the site with 65,000 requests I had a match.\\
	        
	        Once I had a match I logged in. On logging in I was instantly redirected to another page with the message reading "Oops to slow.". I figured the message must have been in the first page that I visited after logging but wasn't there long enough for the page to render in the browser. I ended up using postman to send the authentication request to the site and was able to see the initial response with out being redirected. This had the hack the box code in the H1 tag.\\
	        
	        \subsubsection{Challenge 3: HDC}
	        The final challenge I decided to do was HDC. In this challenge the prompt states that they believe there are some individuals who are up to no good using a website. My goal is to find out who is doing the shady business and send them an email using the web page. This challange was worth 30 points.\\
	        
	        Upon opening the page I am asked to authenticate. I inspected the web page and noticed that the web page has two scripts in it. A legit looking Jquery file as well as a myscript.js. I looked at the myscript file first which really didn't contain any thing of interest. Next, I looked at the web page to see what was happening on submit of the form. I noticed there was a doProcess() action on that form but no doProcess() function in the myscrip. So I opened the Jquery file and searcher for doProcess() and sure enough there was the login action. Lucky for me the login action also include the user credentials. With the users credentials in hand I logged in.\\
	        
	        Once logged in I found an innocuous looking business site that surely didn't look legitimate. I started inspecting the pages for scripts and other clues to what the site might be for. Finally, I started click around to the various different sections of the page and started inspecting the page. Initially, I found nothing of real interest. There was a send email page that seemed a bit odd to me as people usually send emails from an email client. I started looking into this page and found that there was an image on this page. I then looked at this image in another tab to see what the path of the image was. On doing this I found a secret area path that wasn't listed on the home page.\\
	        
	        When I went to the index of the secret area path I found two files. A text file and the image I had previously been looking at. I looked at the text file and found a list of emails. One of the things in the prompt was to see if I could find the person responsible for the site was to use the sites functionality to send an email. So I returned to the email page with a list of emails in had and started sending emails. On the last email that I sent rather than refreshing with a blank for the page refreshed with a response and the tag for the Hack the Box Challenge.\\
	        
	        \section{Completed challenges}
	        Below is the list of challanges I have compleated along with the point totals for each challange.\\
	            \begin{center}
	                \begin{tabular}{ |c|c| }
	                     \textbf{Challenge} & \textbf{Points}  \\
	                     Snake & 10 Points \\
	                     Lernaean & 20 Points \\
	                     HDC & 30 Points \\
	                     \textbf{Total} & \textbf{60 Points}
	                \end{tabular}
	            \end{center}
		
		\end{document}